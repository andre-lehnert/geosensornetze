\chapter{Einf\"uhrung}
\setcounter{page}{1}

%Soll ein Gebäude aufgrund einer Gefahrensituation evakuiert werden, so geschieht dies übli-cherweise über Hinweisschilder, die den Weg zum nächsten Notausgang weisen. Sobald jedoch der Weg zu einem der Notausgänge blockiert ist, kommt es zu unerwünschten Situationen. Eine dynamische Gebäudeevakuierung basierend auf mobilen Geräten und lokaler Kommunikation kann Abhilfe schaffen.  
%
%In dieser Aufgabe soll eine NetLogo-Umgebung zur Simulation einer dynamischen Gebäude-evakuierung entwickelt werden. Dazu sollen Agenten (Menschen mit mobilen Geräten) model-liert werden, die sich frei im Gebäude bewegen, Gefahrensituationen in ihrer Umgebung erfas-sen und benachbarte Agenten warnen. Notausgänge sollen als statische Sensorknoten model-liert werden, die Informationen über ihre Position und Passierbarkeit im Netz verteilen. Zur Lo-kalisierung der mobilen Endgeräte soll ein dezentraler Lokalisierungsalgorithmus implementiert werden. 
%Die Simulation sollte sowohl in einem abstrahierten Grundriss (keine Wände, zufällig verteilte Notausgänge), als auch in einem realen Grundriss (IKG Etage, Lichthof) durchgeführt werden

asdasdasd\cite{Amundson.}


asdasd

asdasdasd\cite{Jonathan.2004}

\section{Aufgabenbeschreibung}

Mit Hilfe der NetLogo-Simulationsumgebung \cite{Netlogo.1999}

