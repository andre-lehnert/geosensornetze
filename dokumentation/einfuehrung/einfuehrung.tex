\chapter{Einf\"uhrung}
\setcounter{page}{1}

Mit Hilfe der NetLogo-Simulationsumgebung \cite{Netlogo.1999} wird eine dynamische Evakuierung von Personen in Gebäuden implementiert. Dazu werden die Grundrisse der Gebäude oder Etage in die Simulationsumgebung geladen. Diese dient als Grundlage für die Platzierung von Personen, Gefahrenevents und Notausgängen. 

Auf der Flucht vor Gefahrenevents werden die Personen von mobilen Geräten unterstützt, die zur Warnung anderer Personen und zur Lokalisierung der Notausgänge dienen.


%Soll ein Gebäude aufgrund einer Gefahrensituation evakuiert werden, so geschieht dies übli-cherweise über Hinweisschilder, die den Weg zum nächsten Notausgang weisen. Sobald jedoch der Weg zu einem der Notausgänge blockiert ist, kommt es zu unerwünschten Situationen. Eine dynamische Gebäudeevakuierung basierend auf mobilen Geräten und lokaler Kommunikation kann Abhilfe schaffen.  
%
%In dieser Aufgabe soll eine NetLogo-Umgebung zur Simulation einer dynamischen Gebäude-evakuierung entwickelt werden. Dazu sollen Agenten (Menschen mit mobilen Geräten) model-liert werden, die sich frei im Gebäude bewegen, Gefahrensituationen in ihrer Umgebung erfas-sen und benachbarte Agenten warnen. Notausgänge sollen als statische Sensorknoten model-liert werden, die Informationen über ihre Position und Passierbarkeit im Netz verteilen. Zur Lo-kalisierung der mobilen Endgeräte soll ein dezentraler Lokalisierungsalgorithmus implementiert werden. 
%Die Simulation sollte sowohl in einem abstrahierten Grundriss (keine Wände, zufällig verteilte Notausgänge), als auch in einem realen Grundriss (IKG Etage, Lichthof) durchgeführt werden

%asdasdasd\cite{Amundson.}

%;asdasdasd\cite{Jonathan.2004}

\section{Aufgabenbeschreibung}

Die Aufgabe besteht in der Umsetzung einer geeigneten Simulationsumgebung (siehe Kapitel \ref{cha:simulationsumgebung}). Auf deren Basis Algorithmen zur Lokalisierung und der Bestimmung eines Fluchtweges zu den Notausgängen entwickelt werden (siehe Kapitel \ref{cha:algorithmik}). Schließlich wird eine Evaluation der Algorithmen in Punkto Effizienz und Zuverlässigkeit durchgeführt und Reflektiert (siehe Kapitel \ref{cha:evaluation}).

Personen werden in der NetLogo-Umgebung als Agenten realisiert, die sich entsprechend eines Bewegungsmodells (siehe Abschnitt \ref{sec:bewegungsmodell}) innerhalb des Grundrisses bewegen. Die initiale Platzierung geschieht zufällig, analog zur Platzierung der Gefahrenevents.\\
Personen besitzen die Fähigkeit diese Gefahrenevents in ihrer Umgebung wahrzunehmen und als Gefahrensituation zu deuten. Der Personen versuchen daraufhin den besten Weg zu einem Notausgang zu finden und benachbarte Personen dabei über ihre mobilen Geräte zu warnen.

Als Gefahrensituationen (siehe Abschnitt \ref{sec:gefahrensituationen}) zählt eine gewisse Anzahl von Giftgasbomben mit eingebautem Zeitzünder, die je ein Gefahrenevent darstellen. Die freigesetzten Gasmengen sind regulierbar und breiten sich innerhalb des freien Raumes aus.

Zur Evakuierung der Personen aus dem Gefahrenevent werden Notausgänge (siehe Abschnitt \ref{sec:notausgaenge}) im Grundriss platziert, deren Position sich während der Simulation nicht ändert, sogenannte \emph{anchor notes}.

Eine feste Position ist notwendig zur Realisierung der dezentralen Lokalisierungsalgorithmen, die auf den mobilen Geräten der Personen aktiv sind und bei einer dynamischen Evakuierung assistieren. Durch eine lokal eingeschränkte Kommunikationsfähigkeit (siehe Abschnitt \ref{sec:kommunikationsmodell}) werden Informationen über die Passierbarkeit der Notausgänge an die mobilen Geräte verteilt. Dies ermöglicht die sichere Evakuierung, falls beispielsweise das Giftgas einen Notausgang erreicht hat oder die Fluchtwege blockiert sind.

