\chapter{Simulationsumgebung}
\label{cha:simulationsumgebung}


\section{Benutzerschnittstelle}
\label{sec:benutzerschnittstelle}


\section{Notausgänge}
\label{sec:notausgaenge}
%Notausgänge: Notausgänge sollen als statische Sensorknoten modelliert werden, die ihre Po-sition sowie Information zu ihrer Passierbarkeit im Netz verteilen. Die Positionen dieser sog. anchor nodes sollen zur Positionierung der mobilen Sensorknoten verwendet werden (Algorithmik). Während einer Simulation sollte der Ausfall einzelner Notausgänge simuliert werden.

\section{Bewegungsmodell}
\label{sec:bewegungsmodell}
%Bewegungsmodell: Die Agenten sollen sich nach einem „Random Walk“ Modell bewegen. Agenten können Gefahren in ihrer Umgebung wahrnehmen und die Information an ihre nächs-ten Nachbarn weitergeben. Sobald ein Agent Informationen über eine Gefahrensituation hat, soll er sich auf dem kürzesten Weg zum nächsten Notausgang bewegen.  Hat ein Agent die Umgebung eines Notausgangs erreicht, soll er aus der Simulation genommen werden.

\section{Gefahrensituationen}
\label{sec:gefahrensituationen}
%Gefahrenevents: Gefahrenevents erscheinen an zufälligen Orten im Grundriss. Ihr Erscheinen sollte zur vollständigen Evakuierung des Areals führen.

\section{Kommunikationsmodell}
\label{sec:kommunikationsmodell}
%Kommunikationsmodell: Als Kommunikationsmodell können beliebige Modelle der Vorlesung implementiert werden. Dabei sollte insbesondere auf die dynamische Netzwerktopologie geach-tet werden. 