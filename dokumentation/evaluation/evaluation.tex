\chapter{Evaluation}
\label{cha:evaluation}
%Eine Evaluation bzgl. Kommunikationsreichweiten, Anzahl „Agenten“ im Gebäude, oder Zeit bis zu vollständiger Evakuierung, etc. soll durchgeführt werden. Insbesondere sollte in der Evaluie-rung auf die Qualität der Lokalisierung in Abhängigkeit der Anzahl Notausgänge / Anchor Nodes, Anzahl Sensorknoten und der Kommunikationsreichweite eingegangen werden. 

\section{Lokalisierung}

Um die beste Lokalisierung mit dem \emph{Multilateration Algorithmus} zu erreichen, spielt die Parameter-Konfiguration die größte Rolle. In den folgenden Tabellen und Analysen wird deutlich, wie groß die Unterschiede zwischen den verschiedenen Ausführungen sein kann, wenn man einen Parameter verändert. \par
Zwischen den Lokalisierungen in einer Zeile werden die Personen nicht neu generiert, die Werte in einer Zeile sind also vergleichbar. Zwischen verschiedenen Tabellen werden die Personen jedoch neu generiert, also kann es zu generellen Abweichungen kommen, da die Verteilung der Personen auch eine Rolle in der Genauigkeit spielt.

\paragraph{Beschreibung der Parameter und Tabellenspalten}

\begin{description}
   \item[number-of-exits] (Tabellenspalte \emph{\#Exits}) \hfill \\
	Anzahl der Ausgänge und Sensorpunkte für den Algorithmus. Der Algorithmus soll mit mehr Sensorpunkten besser approxmieren, ab 3 ist eine realistische Berechnung überhaupt erst möglich.
   \item[personCount] (Tabellenspalte \emph{\#Person}) \hfill \\
	Anzahl der Personen in der Simulation. Eine Mindestanzahl ist nötig, um das Kommunikationsnetz aufzubauen. Mehr Personen bedeutet eine genauere Abschätzung der Distanzen zu den Sensorpunkten, da der Hop Counts zuverlässiger ist. Das ist durch die Verteilung der Personen gegeben, bei sehr vielen Personen ist die Chance höher, dass die Hop Counts \emph{normaler} verteilt sind.
   \item[detection-radius] (Tabellenspalte \emph{detRad}) \hfill \\
	Radius mit dem der UDG bestimmt wird. Das Attribut ist auch für den Hop Count verantwortlich, wenn man ihn höher wählt, dann ist der Hop Count für weiter entfernt stehende Personen kleiner, da der Kommunikationsradius höher ist. Ein kleinerer Wert könnte dazu führen, dass das Netz nicht vollständig alle Notausgänge und Personen verbindet, zu groß führt zu einem ungenauen Hop Count.
   \item[locate-iterations] (Tabellenspalte \emph{\#iter}) \hfill \\
	Wie viele Iterationen der Algorithmus durchläuft. Mehr Iterationen bedeutet höhere Genauigkeit, aber auch höherer Rechenaufwand.
   \item[approx-dist] (Tabellenspalte \emph{~dist}) \hfill \\
	Die Distanz die der Algorithmus zu jedem Hop Count annimmt. Die Zahl ist abhängig von der Personendichte und dem detection-radius.
   \item[average-estimated-distance] (Tabellenspalte \emph{avgDist}) \hfill \\
	Durchschnittlicher Abstand aller geschätzten Positionen zu der tatsächlichen Position. 
   \item[min-estimated-distance] (Tabellenspalte \emph{minDist}) \hfill \\
	Kleinste Distanz von einer geschätzten Position zu einer genauen Position, d.h. die am besten geschätzte Position.
   \item[max-estimated-distance] (Tabellenspalte \emph{maxDist}) \hfill \\
	Größte Distanz von einer geschätzten Position zu einer genauen Position, d.h. die am schlechtesten geschätzte Position.
\end{description}

\begin{table}[h]
\begin{tabular}{|l|l|l|l|l|l|l|l|l|}
\hline
No & \#Exits & \#Person & detRad & \#iter & ~dist & avgDist & minDist & maxDist \\ \hline
1  & 3       & 100      & 50     & 5      & 15    & 52,13   & 10,22   & 126,51  \\ \hline
2  & 3       & 100      & 50     & 5      & 30    & 35,23   & 3,33    & 115,85  \\ \hline
3  & 3       & 100      & 50     & 5      & 35    & 38,65   & 3,24    & 121,78  \\ \hline
4  & 3       & 100      & 50     & 10     & 15    & 52,87   & 10,56   & 126,47  \\ \hline
5  & 3       & 100      & 50     & 10     & 30    & 34,44   & 4,69    & 105,88  \\ \hline
6  & 3       & 100      & 50     & 10     & 35    & 28,18   & 1,84    & 109,89  \\ \hline
7  & 3       & 100      & 50     & 20     & 15    & 52,49   & 10,96   & 126,95  \\ \hline
8  & 3       & 100      & 50     & 20     & 30    & 33,75   & 4,44    & 90,54   \\ \hline
9  & 3       & 100      & 50     & 20     & 35    & 22,95   & 1,81    & 82,22   \\ \hline
10 & 3       & 100      & 50     & 30     & 15    & 52,82   & 10,91   & 126,36  \\ \hline
11 & 3       & 100      & 50     & 30     & 30    & 33,21   & 4,45    & 80,27   \\ \hline
12 & 3       & 100      & 50     & 30     & 35    & 20,1    & 1,17    & 66,19   \\ \hline
\end{tabular}
\caption{Evaluierung für raumplan.png mit 3 Exits und 100 Personen}
\label{fig:eva01}
\end{table}

An der Tabelle \ref{fig:eva01} kann man bei den Anzahl der Iterationen sehen, wie die Genauigkeit des Algorithmus mit der Anzahl der Iterationen steigt. Zwischen 5 und 20 Iterationen ist eine Verbesserung von 50\% zu sehen, wenn die approx-dist dementsprechend gewählt wurde. Von 20 auf 30 Iterationen ist die Verbesserung in Relation mit dem Rechenaufwand wohl irrelevant. In den zukünftigen Experimenten sei die Anzahl der Iterationen stets 20.


\begin{table}[h]
\begin{tabular}{|l|l|l|l|l|l|l|l|l|}
\hline
No & \#Exits & \#Person & detRad & \#iter & ~dist & avgDist & minDist & maxDist \\ \hline
13 & 3       & 150      & 30     & 20     & 15    & 41,15   & 2,57    & 88,11   \\ \hline
14 & 3       & 150      & 30     & 20     & 17    & 35,55   & 4,45    & 72,51   \\ \hline
15 & 3       & 150      & 30     & 20     & 19    & 27,51   & 4,23    & 52,65   \\ \hline
16 & 3       & 150      & 30     & 20     & 20    & 23,46   & 2,37    & 58,46   \\ \hline
\end{tabular}
\caption{Evaluierung für raumplan.png mit 3 Exits und 150 Personen}
\label{fig:eva02}
\end{table}

An der Tabelle \ref{fig:eva02} wird der Effekt von verschiedenen approx-dist Werten veranschaulicht. Man findet eine gute Approximation bei \( \frac{2}{3} \) vom Detection Radius. Wenn die approx-dist noch höher gewählt wird, fallen zu viele geschätzte Position außerhalb der Simulationsumgebung an und das Ergebnis wird verfälscht. Dabei muss noch beachtet werden, dass der detection-radius gesenkt werden konnte, da nun mehr Personen vorhanden sind, die im Kommunikationsnetz den Hop Count genauer weiterreichen und repräsentieren können.


\begin{table}[h]
\begin{tabular}{|l|l|l|l|l|l|l|l|l|}
\hline
No & \#Exits & \#Person & detRad & \#iter & ~dist & avgDist & minDist & maxDist \\ \hline
17 & 3       & 200      & 25     & 20     & 15    & 21,56   & 1,25    & 62,25   \\ \hline
18 & 3       & 200      & 25     & 20     & 17    & 14,59   & 1,64    & 36,32   \\ \hline
19 & 3       & 200      & 25     & 20     & 19    & 22,14   & 2,31    & 66,81   \\ \hline
\end{tabular}
\caption{Evaluierung für raumplan.png mit 3 Exits und 200 Personen}
\label{fig:eva03}
\end{table}

Bei der Tabelle \ref{fig:eva03} wird verdeutlicht, dass bei einer größeren Anzahl von Personen eine höhere Dichte entsteht und durch einen kleineren detection-radius zusammen mit einer passenden approx-dist wesentlich genauere Positionen gefunden werden können. Hier ist auch erneut zu sehen, dass \( \frac{2}{3} \) vom Detection Radius wohl ein sinnvoller Wert ist. In den folgenden Experimenten bleiben wir bei 200 Personen mit einem detection-radius von 25.

\begin{table}[h]
\begin{tabular}{|l|l|l|l|l|l|l|l|l|}
\hline
No & \#Exits & \#Person & detRad & \#iter & ~dist & avgDist & minDist & maxDist \\ \hline
20 & 6       & 200      & 25     & 20     & 15    & 21,72   & 0,65    & 48,62   \\ \hline
21 & 6       & 200      & 25     & 20     & 17    & 12,78   & 1,14    & 35,02   \\ \hline
22 & 6       & 200      & 25     & 20     & 19    & 10,17   & 0,76    & 57,57   \\ \hline
\end{tabular}
\caption{Evaluierung für raumplan.png mit 6 Exits und 200 Personen}
\label{fig:eva04}
\end{table}

In Tabelle \ref{fig:eva04} wird deutlich, wie wichtig es ist viele gleich verteilte Orientierungspunkte für den Algorithmus zu haben. Die Genauigkeit ist im Mittel deutlich höher, insbesondere der kleinste und größte Abstand ist wesentlich besser.

\begin{table}[h]
\begin{tabular}{l|l|l|l|l|l|l|l|l|}
\cline{2-9}
No & \#Exits & \#Person & detRad & \#iter & ~dist & avgDist & minDist & maxDist \\ \cline{2-9} 
23 & 9       & 200      & 25     & 20     & 15    & 18,35   & 1,64    & 42,25   \\ \cline{2-9} 
24 & 9       & 200      & 25     & 20     & 17    & 11,08   & 0,78    & 28,35   \\ \cline{2-9} 
25 & 9       & 200      & 25     & 20     & 19    & 18,03   & 1,25    & 41,47   \\ \cline{2-9} 
\end{tabular}
\caption{Evaluierung für raumplan.png mit 9 Exits und 200 Personen}
\label{fig:eva05}
\end{table}

Bei Tabelle \ref{fig:eva05} mit 9 Notausgängen ist noch eine Verbesserung zu sehen, insbesondere wieder in der min- und max-distance. Bei mehr Orientierungspunkten, einem dichteren Netzwerk und mehr Personen wäre eine Genauigkeit auf wenige Patches genau problemlos möglich.

\section{Evakuierungsdauer}
Um die Evakuierungsdauer zu evaluieren wird ein Szenario erstellt, bei dem sich 100 oder 200 Personen auf der IKG Etage befinden mit 20 Gasbomben. Diese Gasbomben sind alle nicht tötlich, aber sie sind sichtbar, das heißt die Personen wissen, wenn sie losgegangen sind. Wir schauen, wie schnell die Personen von der Etage flüchten können. Die Vorrausetzungen sind, dass ein vollständiger UDG gegeben ist, das heißt bei 100 Personen wird ein detection-radius von 35 angenommen, bei 200 einer von 25, weil der in ausreichend Fällen einen vollständigen Graphen im Ausgangszustand erzeugt. Es wird mit 3, 6 und 9 Notausgängen geprüft. Es soll dabei überprüft, wie schnell die Personen sich mit Hilfe des Kommunikationsnetzes, dem Zellulären Automaten und der Notausgänge retten können. Die verschiedenen Personen und Notausgänge sowie die verschiedenen Anordnungen der Personen stellen dabei den natürlich Arbeitsablauf in einem Bürogebäude dar und machen die Evaluierungsergebnisse interessant und relevant.

\begin{table}[h]
\begin{tabular}{c|c|c|c|c|c|c|c|c|c|c|c|c|}
\cline{2-13}
                           & \multicolumn{6}{|c}{100 Personen}                                                          & \multicolumn{6}{|c|}{200 Personen}                                                          \\ \cline{2-13} 
                           & \multicolumn{2}{|c}{3 Exits} & \multicolumn{2}{|c}{6 Exits} & \multicolumn{2}{|c}{9 Exits} & \multicolumn{2}{|c}{3 Exits} & \multicolumn{2}{|c}{6 Exits} & \multicolumn{2}{|c|}{9 Exits} \\ \hline
\multicolumn{1}{|c|}{\#1}  & 44           & 298           & 18           & 174           & 15           & 166           & 19           & 226           & 18           & 171           & 16            & 160           \\ \hline
\multicolumn{1}{|c|}{\#2}  & 40           & 217           & 16           & 146           & 20           & 140           & 20           & 209           & 13           & 130           & 13            & 142           \\ \hline
\multicolumn{1}{|c|}{\#3}  & 45           & 246           & 18           & 143           & 14           & 138           & 18           & 207           & 16           & 163           & 10            & 122           \\ \hline
\multicolumn{1}{|c|}{\#4}  & 56           & 232           & 15           & 152           & 13           & 144           & 20           & 213           & 15           & 128           & 14            & 169           \\ \hline
\multicolumn{1}{|c|}{\#5}  & 35           & 305           & 16           & 135           & 16           & 180           & 22           & 206           & 13           & 140           & 14            & 135           \\ \hline
\multicolumn{1}{|c|}{\#6}  & 41           & 301           & 18           & 145           & 18           & 126           & 17           & 203           & 14           & 150           & 17            & 120           \\ \hline
\multicolumn{1}{|c|}{\#7}  & 36           & 205           & 19           & 185           & 24           & 153           & 16           & 218           & 15           & 141           & 12            & 134           \\ \hline
\multicolumn{1}{|c|}{\#8}  & 39           & 190           & 15           & 165           & 12           & 137           & 16           & 191           & 14           & 130           & 13            & 147           \\ \hline
\multicolumn{1}{|c|}{\#9}  & 49           & 260           & 16           & 135           & 16           & 155           & 22           & 223           & 11           & 168           & 10            & 150           \\ \hline
\multicolumn{1}{|c|}{\#10} & 51           & 240           & 18           & 175           & 18           & 165           & 15           & 250           & 18           & 151           & 9             & 147           \\ \hline
\multicolumn{1}{|c|}{avg}  & 43,6         & 249,4         & 16,9         & 155,5         & 16,6         & 150,4         & 18,5         & 214,6         & 14,7         & 147,2         & 12,8          & 142,6         \\ \hline
\end{tabular}
\end{table}

Die Tabelle \ref{fig:data} zeigt die Ergebnisse der Simulation unter den gegeben Randbedingungen. Dabei ist die linke Spalte unter jeder Angabe der Anzahl der Exits die Ticks bis zur ersten geretteten Person, die rechte Spalte die Anzahl der vergangenen Ticks bis alle Personen gerettet wurden.

Besonders interessant an den Daten ist der Unterschied zwischen den Personen, obwohl die Anzahl der Personen verdoppelt wird, wird die Flucht aller Personen im Mittel schneller, da die Warnung im Netz schneller verbreitet wird. Das mehr Notausgänge bei einer Flucht helfen ist gegeben, jedoch ist anzumerken, das der Unterschied zwischen 6 und 9 Notausgänge nicht sehr hoch ist. Auch interessant ist bei wenigen Notausgängen und wenigen Personen die stärkeren Schwankungen zwischen maximaldauer und minimaldauer der Evakuierungen, die auf die Gegebenheit der zufälligen Bewegungen der Personen sowie des Netzen zurückzuführen sind.

\section{Fazit}

Vor der Implementierung der Simulation stellten sich einige Fragen um das Thema: Kann mit einem Geosensornetz die Evakuierung eines Gebäudes beschleunigt werden? Welche Faktoren spielen da eine Rolle? Was passiert, wenn Ereignisse über ein Netzwerk dezentral verschickt werden, kann zuverlässig sichergestellt werden, das eine sichere Evakuierung trotzdem möglich ist? Ist es möglich anhand von Orientierungspunkten eine Position von Personen mit einem mobilen Gerät zuverlässig zu bestimmen? Diese Fragen wurden von der Simulation beantwortet; das Geosensornetz zeigte, das selbst Personen die nicht unmittelbar in Gefahr schwebten durch das Netz gerettet werden konnten.

Leider stellten sich aber auch viele Probleme auf, man benötigt sehr viele Personen und Orientierungspunkte um eine sichere und zuverlässige Lokalisierung klarzustellten und selbst dann können soetwas wie Sammelplätze und Leerräume diese ermittelten Positionen verschlechtern oder gar unnütz machen. Aber es gebe einige Szenarien, wie eine Messe, wo Menschen verteilt genug wären so das eine Lokalisierung sowie ein dichtes Netzwerk gegeben wären. Somit hat die Simulation ihre Aufgabe erfüllt und das Szenario realistisch widergegeben.

\section{Ausblick}
Es gibt einige Verbesserungsmöglichkeiten und Erweiterungen, die die Simulation realistischer und Aussagekräftiger gestalten würde. Andere Algorithmen zur Flucht und Lokalisierung wäre möglich, neue Szenarien wie die Cebit wäre sinnvoll um verschiedene Gegebenheiten der Gelände zu simulieren. Man könnte anhand von echten Evakuierungsdaten das Verhalten der Personen realistischer gestalten und andere Gefahrenszenarios einführen. Im Folgenden wird eine weitere Idee für einen Orientierungsalgorithmus demonstriet.

\subsubsection{Alternativer Orientierungsalgorithmus}
\label{sec:bug-0}
Als alternativer Orientierungsalgorithmus zur lokalen Fluchtwegfindung kann der \emph{Bug-0-Algorithmus} dienen.

